\documentclass[color,t,presentation,english,aspectratio=169]{beamer}



\usetheme[
% insert a header below title showing the sections (short section names) for navigation,
	%secheader,
	%
% insert a header above title showing current chapter ou section 
	%currentsecheader,
	%
% text is in english, default is french
	isinenglish,
	%
% print outline at beginning section, default is false	
	outlineabs,
	%
% print partner's log: ubl, ccbyncsa, cnrs, subatech, irisa, inserm, lab-sticc, marsouin, ls2n, gepea, ubl, loustic, ifremer
% TO DO: lego, atol, latim, cominglab, etc.
	secondlogo=ccbyncsa,
	%
% print notes, default is false	
	%printnotes,
	%
% slides per page: 1 (default), 2, 4 --landscape--	
	slidesparpage=1,
	%
% section or chapter in headline, default chapter	
	sectorchapt=section, %section
	%
% Add a footline with title and author
	% footline,
	%
	]
{IMTAtlantique}




\usepackage[utf8]{inputenc}
\usepackage[english]{babel} 
%\usepackage[latin1]{inputenc}
\usepackage{xspace}
\usepackage{amssymb}
%\usepackage{amsmath}
\usepackage{graphicx}
\usepackage{color}
\usepackage{xcolor}
\usepackage{multirow}
\usepackage{rotating}
\usepackage{array}%
\usepackage{fancybox}
\usepackage{hhline}
\usepackage{tikz}
\usetikzlibrary{calc,arrows,shapes,decorations}
\usepackage{xifthen}
\usepackage{bbm}
\usepackage[absolute]{textpos}



\usepackage{pgfpages}
\usepackage{format-IMT-Atlantique}

%
% Properly spaced abbreviations
% Taken from the CVPR's style package (https://stackoverflow.com/a/39363004)
%
\usepackage{xspace}

% Add a period to the end of an abbreviation unless there's one
% already, then \xspace.
\makeatletter
\DeclareRobustCommand\onedot{\futurelet\@let@token\@onedot}
\def\@onedot{\ifx\@let@token.\else.\null\fi\xspace}
%
\def\eg{\emph{e.g}\onedot} \def\Eg{\emph{E.g}\onedot}
\def\ie{\emph{i.e}\onedot} \def\Ie{\emph{I.e}\onedot}
\def\cf{\emph{c.f}\onedot} \def\Cf{\emph{C.f}\onedot}
\def\etc{\emph{etc}\onedot} \def\vs{\emph{vs}\onedot}
\def\wrt{w.r.t\onedot} \def\dof{d.o.f\onedot}
\def\etal{\emph{et al}\onedot}
\makeatother



\usepackage{fontspec}
\setmainfont{Arial}

\setbeamertemplate{section in head/foot shaded}[default][75]
\setbeamertemplate{section in head/foot shaded}[default][25]


\usepackage[acronym]{glossaries}
\loadglsentries{glossary}
\makenoidxglossaries


\usepackage{layouts}
\usepackage{printlen}
\usepackage{verbatim}

\makeatletter
\newcommand{\printfontsize}[1]{{#1\the\dimexpr\f@size pt\relax}}
\makeatother

\usepackage[style=ieee]{biblatex}
\addbibresource{references.bib}

\title{\vspace{3\baselineskip}What if attackers are indeed inside?\\}

\subtitle{A Systematic Analysis of Label-flipping Attacks against Federated Learning for Intrusion Detection\\}

\author{\textbf{Léo Lavaur}\textsuperscript{1}, Yann Busnel\textsuperscript{2}, and Fabien Autrel\textsuperscript{1}}

\institute{\textsuperscript{1} IMT Atlantique, \textsuperscript{2} IMT Nord Europe\\}
    

\date{ARES (BASS) 2024, August 2, 2024}

\begin{document}

\maketitle

% \begin{frame}
% 	\frametitle{\acrfull{nids}}
% 	\glsunset{nids}

% 	\begin{figure}
% 		\centering
% 		\includegraphics[width=\linewidth]{./figures/mlp-workflow.pdf}
% 		\caption{Typical \acrshort{nids} workflow.}
% 	\end{figure}

% 	\begin{itemize}
% 		\item Great performance with \Gls{dl} (on public datasets at least (: )
% 		\item \textbf{Limitations}: lack of labelled data, risk of local bias or skewed data distribution, inefficient against new attacks.
% 	\end{itemize}

% \end{frame}

\begin{frame}
	\frametitle{\acrfull{fids}}

	\begin{columns}
		\begin{column}{0.4\linewidth}
			\vspace{-\textheight}
			\begin{itemize}
				\item \Acrfull{fl} is a distributed \Acrfull{ml} paradigm.
				\item Can train a global model without sharing local data.
				\item Can be used in \Acrfull{cids}:
				\begin{itemize}
					\item Extend the training data
					\item Effectively share knowledge (\eg, on specific classes, instances) between participants
				\end{itemize}
			\end{itemize}
		\end{column}
		\begin{column}{0.6\linewidth}
			\begin{overlayarea}{\linewidth}{\textheight}
				\only<1>{\includegraphics[width=\linewidth]{./figures/fl-1.pdf}}
				\only<2>{\includegraphics[width=\linewidth]{./figures/fl-2.pdf}}
				\only<3>{\includegraphics[width=\linewidth]{./figures/fl-3.pdf}}
				\only<4>{\includegraphics[width=\linewidth]{./figures/fl-4.pdf}}
				\only<5>{\includegraphics[width=\linewidth]{./figures/fl-5.pdf}}
			\end{overlayarea}
		\end{column}
	\end{columns}

\end{frame}
\glsunset{fids}
\glsunset{fl}
\glsunset{ml}

\begin{frame}
	\frametitle{\Gls{fl} against malicious contributions}

	\begin{columns}
		\begin{column}{0.4\linewidth}
			\vspace{-\textheight}
			\begin{itemize}
				\item \Gls{fl} is highly susceptible to poisoning.
				\begin{itemize}
					\item Few studies on their impact in \gls{fids}.
				\end{itemize}
			\end{itemize}
		\end{column}
		\begin{column}{0.6\linewidth}
			\begin{overlayarea}{\linewidth}{\textheight}
				\only<1>{\includegraphics[width=\linewidth]{./figures/fl-poisoning-1.pdf}}
				\only<2>{\includegraphics[width=\linewidth]{./figures/fl-poisoning-2.pdf}}
				\only<3>{\includegraphics[width=\linewidth]{./figures/fl-poisoning-3.pdf}}
				\only<4>{\includegraphics[width=\linewidth]{./figures/fl-poisoning-4.pdf}}
				\only<5>{\includegraphics[width=\linewidth]{./figures/fl-poisoning-5.pdf}}
			\end{overlayarea}
		\end{column}
	\end{columns}

\end{frame}

\begin{frame}
\frametitle{Types of poisoning attacks}
\begin{itemize}
    \item By component:
    \begin{itemize}
			\item Data poisoning (\eg, \textbf<2>{label-flipping}, clean-label attacks, backdoors)
			\item Model poisoning (\eg, gradient boosting, noising)
    \end{itemize}
		\item By target:
		\begin{itemize}
			\item \textbf<2>{Untargeted}: affect the model's global performance
			\item \textbf<2>{Targeted}: modify its behavior on specific classes or instances
		\end{itemize}
		\item By frequency:
		\begin{itemize}
			\item one-shot: attacks are performed once
			\item \textbf<2>{iterative/continuous: at each round}
			\item adaptive: reacts to the model aggregation
		\end{itemize}
	\end{itemize}
\end{frame}

\tableofcontentsimtatlantique

\section{Experiments}

\begin{frame}
	\frametitle{Research questions}
	\begin{itemize}
		\item Research questions (RQ):
    \begin{itemize}
			\item RQ1. Is the behavior of poisoning attacks predictable?
			\item RQ2. Are there beneficial or harmful combinations of hyperparameter under poisoning attacks?
			\item RQ3. Can FL heal itself from poisoning attacks?
			\item RQ4. Are IDS backdoors realistic using label-flipping attacks?
			\item RQ5. Is there a critical threshold where label-flipping attacks begin to impact performance?
    \end{itemize}
\end{itemize}
\end{frame}

\begin{frame}
	\frametitle{Experimental setup}
	\begin{columns}
		\begin{column}{0.5\linewidth}
			\begin{itemize}
				\item Used dataset: \textbf<2>{sampled} NF-V2 version of CSE-CIC-IDS2018
				\begin{itemize}
						\item ports and IP addresses are removed
				\end{itemize}
				\item Same class distribution in the training and testing sets
				\begin{itemize}
						\item 80\% of the dataset is used for training
						\item 20\% of the dataset is used for testing
				\end{itemize}
				\only<3>{%
					\item Assessment of the representativity of the dataset sampling
					\begin{itemize}
						\item Cross-projections of the malicious traffic from two datasets in two dimensions using PCA
					\end{itemize}}
			\end{itemize}
		\end{column}
		\begin{column}{.5\linewidth}
			\only<3>{\begin{figure}
				\centering
				\vspace{-4ex}
				\includegraphics[width=.7\linewidth]{figures/pca-projection.pdf}
				\caption{Cross-projection of the malicious traffic from two datasets in two dimensions using PCA.}
			\end{figure}}
		\end{column}
	\end{columns}

\end{frame}

\begin{frame}
\frametitle{Experimental setup}
\begin{itemize}
    \item A simple multilayer perceptron (MLP) model with two hidden layers is used
    \item Baseline (centralized training)
    \begin{itemize}
        \item F1-score: 0.966
        \item Accuracy: 0.992
    \end{itemize}
    \item FL setup
    \begin{itemize}
        \item Cross-silo setting: all clients are available at each round
        \item The dataset is partitioned into 10 \gls{iid} shards of 80,000 data points
        \item Models are aggregated using \texttt{FedAvg}
    \end{itemize}
    \item Attack model
    \begin{itemize}
        \item Malicious participants can alter their local datasets before training
        \item Data-poisoning: label-flipping attacks (targeted and untargeted)
        \item \Gls{dpr}: ($\alpha$) $\rightarrow$ proportion of flipped samples
        \item \Gls{mpr}: ($\tau$) $\rightarrow$ proportion of attackers
    \end{itemize}
\end{itemize}
\end{frame}

\section{RQ1: Predictability}

\begin{frame}
\frametitle{RQ1: Is the behavior of poisoning attacks predictable?}
\begin{figure}
	\centering
	\includegraphics[width=.8\textwidth]{figures/predictability-all.pdf}
	\caption{Accuracy of the poisoned model by seed.}
\end{figure}
\end{frame}

\begin{frame}
	\frametitle{RQ1: Is the behavior of poisoning attacks predictable?}
	\begin{figure}
		\begin{columns}
			\begin{column}{.2\textwidth}
				\vspace{-\textheight}
				\caption{Predictability depending on the hyperparameters.}
			\end{column}
			\begin{column}{.8\textwidth}
				\includegraphics[width=\linewidth]{figures/predictability-envelopes.pdf}
			\end{column}
		\end{columns}
	\end{figure}
\end{frame}

\begin{frame}
\frametitle{RQ1: Is the behavior of poisoning attacks predictable?}
\centering
\vfill
Answer: \onslide<2>{Nope.}
\vfill
\end{frame}

\section{RQ2: Hyperparameters}

\begin{frame}
	\frametitle{RQ2: Do hyperparameters influence the effect of poisoning attacks?}

	\begin{figure}
		\centering
		\includegraphics[width=.8\textwidth]{figures/hyperparams-continous-cicids-small.pdf}
		\caption{Effect of the hyperparameters on the accuracy of the poisoned model.}
	\end{figure}
\end{frame}

\begin{frame}
	\frametitle{RQ2: Do hyperparameters influence the effect of poisoning attacks?}

	\begin{figure}
		\centering
		\includegraphics[width=\textwidth]{figures/hyperparams-late.pdf}
		\caption{Effect of the hyperparameters on the accuracy of the poisoned model in the \texttt{late} scenario.}
	\end{figure}
\end{frame}


\begin{frame}
\frametitle{RQ2: Do hyperparameters influence the effect of poisoning attacks?}

Answer:
\begin{itemize}
	\item No impact on the average performance
	\item Significant impact on the variance of the results
	\item High batch size leads to more inertia
	\begin{itemize}
		\item The impact is less instantaneous
	\end{itemize}
\end{itemize}
\end{frame}

\section{RQ3: Recovery}

\begin{frame}
	\frametitle{RQ3: Can FL heal itself from poisoning attacks?}
	\begin{figure}
		\centering
		\includegraphics[width=.5\textwidth]{figures/redemption.pdf}
		\caption{Recovery of the model after a poisoning attack.}
	\end{figure}
\end{frame}

\begin{frame}
\frametitle{RQ3: Can FL heal itself from poisoning attacks?}
Answer:
\begin{itemize}
	\item Yes.
	\item Attack stopping $\Longleftrightarrow$ randomize initial parameters
	\begin{itemize}
		\item One round suffises to recover (in our settings).
	\end{itemize}
\end{itemize}
\end{frame}

\section{RQ4: Backdoors}

	\begin{frame}
	\frametitle{Appropriate metrics}
	
	\begin{columns}
		\begin{column}{0.55\textwidth}
			\textbf{\Gls{aasr}}

			\begin{itemize}
				\item Targeted attacks:
				\begin{itemize}
					\item miss rate on a class: $\frac{\text{FN}_\text{class}}{\text{TP}_\text{class} + \text{FN}_\text{class}}$
				\end{itemize}
				\item Untargeted attacks:
				\begin{itemize}
					\item misclassification rate: $1 - \text{accuracy}$
				\end{itemize}
			\end{itemize}

			\textbf{\Gls{rasr}}

			\begin{itemize}
				\item $\frac{
					\text{max}(\text{AASR}_\text{benign}, \text{AASR}_\text{attack}) - \text{AASR}_\text{benign}
				}{
					1 - \text{AASR}_\text{benign}
				}$
			\end{itemize}
		\end{column}
		\begin{column}{0.35\textwidth}
			\vspace{-3ex}
			\begin{figure}
				\centering
				\includegraphics[width=.8\linewidth]{figures/baseline.pdf}
				\caption{Baseline for benign runs.}
			\end{figure}

		\end{column}
	\end{columns}

	

\end{frame}


\begin{frame}

	\frametitle{RQ4: Are IDS backdoors realistic using label-flipping attacks?}
	\begin{figure}
		\centering
		\includegraphics[width=.8\textwidth]{figures/backdoors.pdf}
		\caption{Backdoor success rate.}
	\end{figure}

\end{frame}

\begin{frame}
\frametitle{RQ4: Are IDS backdoors realistic using label-flipping attacks?}

Answer:
\begin{itemize}
	\item Yes, but\dots\only<2>{
	\item The model's generalization capabilities can mitigate the impact
	\begin{itemize}
		\item especially with class overlaps between characteristics
	\end{itemize}
	\item The attack's effectiveness is highly dependent on the target
	\item We need a significant number of attackers.
	}
\end{itemize}
\end{frame}



\section{RQ5: Threshold}

\begin{frame}
\frametitle{RQ5: Is there a critical threshold for label-flipping to be effective?}
\vspace{-2ex}
\begin{figure}
	\centering
	\includegraphics[width=.8\textwidth]{figures/attacks.pdf}
	\vspace{-2ex}
	\caption{Impact of $\tau$ and $\alpha$ on the attack's effectiveness.}
\end{figure}
\end{frame}



\section{Conclusion}

\begin{frame}
\frametitle{Conclusion}

We build a reproducible framework to study the impact of label-flipping attacks in \gls{fids} using \gls{fl}, here are our main findings (yet):

\begin{itemize}
    \item Label-flipping attacks can have a significant impact on the performance of FL models, especially targeted ones
    \item The ASR is closely related to the number of flipped samples overall, which can be approximated in \gls{iid} settings by $\alpha * \tau$
    \item Targeted label-flipping attacks strive on well-detected targets, but can be significantly mitigated by the model’s generalization capabilities
    \item Mitigation strategies must be adapted to the use case specificities (e.g., constrained environments)
\end{itemize}
\end{frame}


\begin{frame}
	\frametitle{Future Work}
	\begin{itemize}
		\item Extend the study to other datasets. \only<2>{\textcolor{red}{[DONE!]}}
		\item Study the impact of the data distribution on the ability to detect attacks. \only<2>{\textcolor{red}{[DONE!]}}
		\item Extend to other feature sets and poisoning attacks.
		\begin{itemize}
			\item Our evaluation framework is generic enough (and open source!) to make extending the results easy.
		\end{itemize}
	\end{itemize}
	\vspace{3\baselineskip}
	\begin{center}
		\only<2>{\emph{
			Want to know about the new results? I defend my thesis on \textbf{October 7th, 2024} at \textbf{14:00} in \textbf{Rennes}. You are welcome to join!
			}}
			
		\end{center}
	\end{frame}
	
	
\end{document}